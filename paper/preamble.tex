% This is stuff that is required for the document to compile. It is not
% related to formatting. 
%\usepackage[dvipsnames]{xcolor} % for Blue; also, must before tikz, pstricks
\usepackage{lipsum}
\usepackage{booktabs}  % for tables
\usepackage{dcolumn}   % for tables
%\usepackage{hhline}   % for tables
\usepackage{multirow}  % for tables
\usepackage{rotating}
\usepackage{xspace}
\usepackage{hyphenat} 
\usepackage{enumerate}
\usepackage{mfirstuc} % captialize first letter
\usepackage{tikz} % two-way onarrow
\usepackage{xparse} % two-way onarrow
% for epoch transaction
\usepackage{listings}
\definecolor{dkgreen}{rgb}{0,.6,0}
\definecolor{dkblue}{rgb}{0,0,.6}
\definecolor{dkyellow}{cmyk}{0,0,.8,.3}
\lstset{
  language        = c,
  basicstyle      = \small\ttfamily,
  keywordstyle    = \bfseries\color{dkblue},
  stringstyle     = \color{red},
}

%\usepackage[breaklinks=true,
%            pdfdisplaydoctitle=true,
%            %pagebackref, % we don't need this for a paper
%            bookmarksnumbered=true,
%            colorlinks = true,
%            citecolor = black, % change this if you want to highlight citations, maybe Blue
%            anchorcolor = black,
%            urlcolor = Blue, % apparently, Blue != blue
%            linkcolor = Blue,
%            pdfborder={0 0 0},
%            pdfpagelabels,
%            pdfpagelayout=SinglePage,
%            hyperfootnotes=false
%            ]{hyperref}

%\usepackage[square,comma,numbers,sort&compress]{natbib}

%\usepackage{ulem}      % for strikethrough and underlining

\usepackage{wrapfig}
\usepackage{textcomp}
\usepackage{tabularx}
\usepackage{pifont}
\usepackage{url}
\def\UrlBreaks{\do\/\do-}
\usepackage{breakurl}

\usepackage{wrapfig}

\usepackage{ulem}
\normalem

% for cap the first letter for a cmd
\usepackage{mfirstuc}

% for getting table 1 in hotnets09 submission to work
\usepackage{colortbl} % colorful columns and rows
\usepackage{array}    % needed for 'b' argument in tabular preamble

\usepackage{dblfloatfix}

\usepackage[font=small]{caption}
\usepackage{subcaption}

\usepackage[noend]{algpseudocode}
\algrenewcomment[1]{\hfill// #1}%
\algrenewcommand\algorithmicindent{1.25em}

\definecolor{darkgreen}{rgb}{0,0.75,0}

% Hack algpseudocode to be more Python-like
\algrenewcommand\algorithmicdo{:}
\algrenewcommand\algorithmicwhile{\textbf{while}}
\algrenewcommand\algorithmicfor{\textbf{for}}
\algrenewcommand\algorithmicforall{\textbf{for all}}
\algrenewcommand\algorithmicloop{\textbf{loop}}
\algrenewcommand\algorithmicrepeat{\textbf{repeat}}
\algrenewcommand\algorithmicuntil{\textbf{until}}
\algrenewcommand\algorithmicprocedure{\textbf{procedure}}
\algrenewcommand\algorithmicfunction{\textbf{function}}
\algrenewcommand\algorithmicif{\textbf{if}}
\algrenewcommand\algorithmicthen{:}
\algrenewcommand\algorithmicelse{\textbf{else}}
\algrenewcommand\algorithmicrequire{\textbf{Require}:}
\algrenewcommand\algorithmicensure{\textbf{Ensure}:}
\algrenewcommand\algorithmicreturn{\textbf{return}}
\algdef{SE}[WHILE]{While}{EndWhile}[1]{\algorithmicwhile\ #1\algorithmicdo}{\algorithmicend\ \algorithmicwhile}%
\algdef{SE}[FOR]{For}{EndFor}[1]{\algorithmicfor\ #1\algorithmicdo}{\algorithmicend\ \algorithmicfor}%
\algdef{S}[FOR]{ForAll}[1]{\algorithmicforall\ #1\algorithmicdo}%
\algdef{SE}[IF]{If}{EndIf}[1]{\algorithmicif\ #1\algorithmicthen}{\algorithmicend\ \algorithmicif}%
\algdef{C}[IF]{IF}{ElsIf}[1]{\algorithmicelse\ \algorithmicif\ #1\algorithmicthen}%
\algnotext{Function}                
\algnotext{EndFunction}
\algnotext{EndFor}
\algnotext{EndIf}
\algnotext{EndWhile}
\algnewcommand\Or{\textbf{or}\xspace}
\algnewcommand\myAnd{\textbf{and}\xspace}

\usepackage{amsmath,amscd}
\usepackage{amssymb}
\usepackage{amsfonts}
\usepackage{amsthm}

\usepackage[noeka]{mathrmletter}
\usepackage{tgtermes}

\newif\ifextended
\newif\iflongbatching
\newif\ifsubmission
\newif\ifelementary
\newif\ifwithdbproof
\newif\ifbuildanonapdx

% peanut gallery comments
% NOTE: Comment out the line below if you want a draft with no red comments.
% NOTE: Commenting out this line may replace some of the red comments with 
%       extra spaces or newlines.
\def\noeditingmarks{}
%


\newcommand{\zdag}{{$^{\dagger}$}\xspace}
\newcommand{\zddag}{{$^{\ddagger}$}\xspace}
\newcommand{\zstar}{{$^{\star}$}\xspace}

\newcommand{\textred}[1]{\textcolor{red}{#1}}
\newcommand{\textblue}[1]{\textcolor{blue}{#1}}
\newcommand{\textgreen}[1]{\textcolor{darkgreen}{#1}}
\ifx\noeditingmarks\undefined
   \newcommand{\pgwrapper}[3]{\begingroup \color{#1} #2: #3 \endgroup}
   \newcommand{\pgwrapperb}[1]{\textbf{#1}}
   \newcommand{\dangerwrapper}[1]{{\color{red}#1}}
\else
   \newcommand{\pgwrapperb}[1]{}
   \newcommand{\pgwrapper}[3]{}
   \newcommand{\dangerwrapper}[1]{}
\fi
\newcommand{\cheng}[1]{\pgwrapper{red}{CT}{#1}}
\newcommand{\myang}[1]{\pgwrapper{blue}{MY}{#1}}
% end peanut gallery comments

\def\hn{\sffamily\selectfont}
\newcommand{\mpfont}{\hn\scriptsize}

\ifx\noeditingmarks\undefined
    \newcommand{\MPworker}[2]{\unskip{\color{#1}\vrule\vrule}{\marginpar{\raggedright\color{#1}\mpfont #2}}}
\else
    \newcommand{\MPworker}[2]{\unskip}
\fi

\newcommand{\CP}[1]{\MPworker{red}{CT: #1}}
\newcommand{\MP}[1]{\MPworker{blue}{MY: #1}}

\ifx\noeditingmarks\undefined
    \newcommand{\changebars}[2]{%
    [{\color{magenta}\em\begingroup{#1}\endgroup}][{\color{magenta}\sout{#2}}]}
\else
    \newcommand{\changebars}[2]{#1}
\fi

\newcommand{\changebarsii}[2]{#1}

\newcommand{\codesize}{\fontsize{8}{9}\selectfont}
 
% ref appendix in different styles, depending on whether it's the
% extended version and depending on what the text is trying to do.

\setlength{\marginparwidth}{15mm}
\setlength{\marginparsep}{0.35mm}

\def\t{\textit}
\def\as{\leftarrow}

\newcommand{\CF}[1]{\xmakefirstuc{#1}}
\newcommand{\heading}[1]{
  \vspace{1ex}
  \noindent
  \textbf{#1}}

